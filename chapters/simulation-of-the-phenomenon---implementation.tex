\chapter{Simulation of the phenomenon - implementation}

\section{Technology and tools}

The project described in this work is written in two parts --- backend and the app.

As we could pick any technology whatsoever we picked Ruby for the backend as this is the backend technology we are most familiar with. We have also picked a web framework called Ruby on Rails as it is the most popular web framework for Ruby. In the long run, it saved us a lot of time that would be required to set up a web server, handle requests, make database queries and many more.

The backend consists of a few REST\footnote{Representational State Transfer} endpoints. One of them receives user's location and runs the algorithm against provided location to determine, whether the input is legitimate or not. After it is calculated, data is preserved in a database with an appropriate label. There is also an endpoint to list locations of all users at the time and an endpoint to return specific user's history.

Implementing a mobile application didn't seem like an easy task at first. Currently, if someone wants to make a mobile application they have to target a specific platform --- Android or iOS. Developing for these platforms isn't easy at all. For the Android development, you have to know Java, Kotlin or C++, have Android Studio installed, and knowledge on what to do. Fortunately, there are many resources on how to start. Authors of this work do not have an Android phone, so there would be issues with running the app in a simulator, which would defeat the purpose of this work. The iOS side of things is not easier either, as it requires knowledge of Swift or Objective-C, having XCode installed on a macOS device, a paid developer account and many more. Even if we decided to go with iOS, it would require a lot of work, time and resources to develop a simple app. At that point, we started reading through the Internet and we learned there are many JavaScript compilers that compile your code into native-to-the-platform code. One of them is React Native --- a technology that allows writing JavaScript code in a React Framework that is compiled to a native app. This made great news to us, as we didn't have to learn any other tools or languages, as we know JavaScript perfectly well.

The app consists of a view that presents a native-to-iOS map. On the map, there are markers that represent locations of all app users in real time. If a marker is held, more details about the location are shown --- a name of the user and the time of the measurement. When a marker is clicked, the map goes into the \textit{history} mode where all historical locations of that particular user are displayed. You can then click anywhere on the map to go back to the \textit{normal} mode. In the background, any time the app is open, current location of the user is sent to the server to update user's location.

\section{Implementation details}
In this section, we will describe two main algorithms used to store locations and retrieve past walking paths of the user.

In our database, we have two tables: one for representing users and one for locations. I will refer to the model of a user with \textit{User} and to a location with \textit{Location}.

The process of storing a new sample consists mainly of computing average speed needed to achieve new location. To compute the distance between to points we use Haversine formula which is described as below:

\begin{equation}
    \displaystyle d=2r\arcsin \left({\sqrt {\sin ^{2}\left({\frac {\varphi _{2}-\varphi _{1}}{2}}\right)+\cos(\varphi _{1})\cos(\varphi _{2})\sin ^{2}\left({\frac {\lambda _{2}-\lambda _{1}}{2}}\right)}}\right)
\end{equation}
where~$\varphi _{1},\varphi _{2}$~are~latitudes~and~$\lambda _{1},\lambda _{2}$~are~longitudes.

Here is some pseudo code of the storing algorithm:
\begin{lstlisting}
my_user = User.find(user_id)
last_location = my_user.locations().last()
distance = Haversine(new_location, last_location)
time = new_location.creation_time() - last_location.creation_time()
speed = distance / time
if (speed > threshold)
    new_location.status = unreachable
else
    new_location.status = reachable
end
return new_location.save()
\end{lstlisting}
Another crucial part of our system is the algorithm rendering old paths. Here you can get a quick glimpse of it:
\begin{lstlisting}
my_user = User.find(user_id)
my_user.locations()
routes = [] // group reachable locations into walking paths
counter = 0

for (location in locations)
    if (location.status == reachable)
        routes[counter].push(location)
    else
        if (routes[counter] != null)
            counter+=1
        end
    end
end

// remove stoppage points
routes.filter do (path)
    // calculating center of mass in path
    com = {lat: 0, lon: 0}

    for (location in path)
        com.lat += location.lat
        com.lon += location.lon
    end

    com.lat /= path.size()
    com.lon /= path.size()

    // check if there is a point in the path
    // further than a threshold from the centre of mass
    return path.any do (location)
        distance = Haversine(location, com)
        return distance > threshold
    end
end
\end{lstlisting}
As you can see we group reachable locations together and after grouping, we check if any of the paths in routes has all of its locations in a small area. This would mean that probably this path is actually a temporary stoppage on traffic lights and we do not want to render it as walking.
