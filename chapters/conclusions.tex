\chapter{Summary}

\section{Conclusions}
Although we had little problems with GPS accuracy, we managed to achieve our goal. This project proves that we are able to distinguish walking modes of the person using only GPS data. There is a lot of room for improvements and tweaks but it is working properly. We experienced how to work with inaccurate data and filter out possible anomalies.

\section{Model in the context of existing solutions}
Applications available today doesn't distinguish walking modes at all. Strava, Endomondo and Google maps can't tell whether you have been running, walking or driving a car. They only track your statistics. We think that our solution could revolutionize outdoor activity tracking apps. Being able to tell how long and where have you been walking in the past is a crucial element of data about your daily activities.

\section{Biggest challenges and achievements}
The biggest challenge we have encountered was creating a reliable algorithm that would accurately filter data that is not coming from a walking human. As can be seen in the chapter~\ref{ch:model}, proposed model is complicated and consists of many parts. Removing points that are apart from each other more than some distance, filtering out points that are in a near approximation that are a result of a \textit{teleportation}, approximating stationary location by averaging coordinates etc. To come up with all of these solutions took quite some time. Another thing was implementing those solutions, which took even more time. We are very happy we came up with a working solution that is working quite reliably and, to us, is very impressive. Should we ever have to implement a location-based mobile app, we would definitely use a solution described in this article. It is not very hard to implement and improves a system heavily.

\section{Future work and possible directions to be taken} \label{sec:futurework}
There are many ways to improve on the work described in this article. Current algorithm does filter out data that seems to be generated by not walking and those that seem to be generated by a computer. The biggest challenge there is is the readout accuracy. Currently, when using simple GPS module embedded in an iPhone there are few-meter differences compared to the actual position. Of course, one way to get a more accurate position of a stationary object is to take multiple readouts and take an average of all points. The problem arises when the object is in motion. During testing, we gathered data that was up to 19 meters inaccurate because of tall buildings on the south side that were obtrusive to the GPS satellites located on the equator (more on that to be read in section \ref{sec:inaccuratereadoutsexample}). This is one thing to be worked on and to be improved. Authors of this article do not have a good way of determining a more accurate location in these cases. Probably an algorithm that would take the environment into consideration could help. Another thing would be to make the path more linear. Although not accurate, we have found that people move in close-to-straight lines. Such approximation could show more accurate results, but not necessarily. It could also lead to results that are off, because of making paths \textit{too} straight. This is definitely something to consider.
